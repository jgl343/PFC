

La plataforma de Android nos ofrece mecanismos para conservar
 la vida de la batería. De forma que nos ofrece diversas APIs para
 planificar los procesos, describimos algunas en la sección de APIs.

Dejando las APIs a un lado, caben destacar  dos funciones de
 ahorro de energía que presenta Android, ya que estas prolongan 
la vida de la batería administrando la forma en que las apps se 
comportan cuando un dispositivo no está conectado a una 
fuente de energía. Descanso reduce el consumo de batería
 aplazando la actividad de CPU y de red de las apps en segundo
 plano cuando el dispositivo no se usa durante períodos de
 tiempo prolongados. App Standby aplaza la actividad de red 
en segundo plano de las apps con las cuales el usuario no haya
 interactuado recientemente.

A continuación vamos a profundizar en ambas funciones.

\paragraph{Modo Descanso:}

Si un usuario deja un dispositivo desconectado, quieto y con la 
pantalla apagada durante un período de tiempo determinado, este 
entra en el modo Descanso. En el modo Descanso, el sistema intenta 
conservar la carga de la batería restringiendo el acceso por parte de
 las apps a servicios de uso intenso de red y CPU. También evita que
 las apps accedan a la red y aplaza sus tareas, sincronizaciones
 y alarmas estándares.

De forma periódica, el sistema desactiva el modo Descanso durante 
un tiempo breve para permitir que las apps completen sus actividades 
aplazadas. Durante este período de mantenimiento, el sistema ejecuta
 todas las sincronizaciones, tareas y alarmas pendientes, y permite 
que las apps accedan a la red.
\begin{figure}[h]
\includegraphics[scale=0.20]{imagenes/doze.png} 
\caption{Descanso proporciona un período de mantenimiento
 recurrente para que las apps usen la red y controlen actividades pendientes.}
\end{figure}



Al finalizar cada período de mantenimiento, el sistema vuelve a activar 
el modo Descanso, suspender el acceso a la red y aplazar tareas, 
sincronizaciones y alarmas. Con el paso del tiempo, el sistema programa
 los períodos de mantenimiento cada vez con menos frecuencia, lo cual
 permite reducir el consumo de batería en casos de inactividad durante 
más tiempo cuando el dispositivo no está conectado a un cargador.

Si bien el usuario activa el dispositivo moviéndolo, encendiendo la pantalla
 o conectándolo a un cargador, el sistema desactiva el modo Descanso 
y todas las apps vuelven a la actividad normal.

\paragraph{Restricciones del modo Descanso:}
Durante el modo Descanso, se aplican las siguientes restricciones a tus apps:
\begin{enumerate}
\item Se suspende el acceso a la red.
\item El sistema ignora los wake locks.
\item Las alarmas estándares AlarmManager (incluidas setExact() y setWindow())
 se aplazan hasta el siguiente período de mantenimiento.
\begin{enumerate}
\item Si necesitas programar alarmas que se activen en el modo Descanso, 
usa setAndAllowWhileIdle() o setExactAndAllowWhileIdle().
\item Las alarmas programadas con setAlarmClock() se activan con normalidad;
 el sistema desactiva el modo Descanso antes de que esas alarmas se activen.
\end{enumerate}
\item El sistema no realiza escaneos de Wi-Fi.
\item El sistema no permite que se ejecuten adaptadores de sincronización.
\item El sistema no permite que se ejecute JobScheduler.
\end{enumerate}

\paragraph{Información sobre App Standby}

El modo App Standby permite que el sistema determine si una app se
 encuentra inactiva cuando el usuario no la usa activamente. El sistema
 hace esta determinación cuando el usuario no aplica toques en la app
 durante un período determinado y cuando no rige ninguna de las 
siguientes condiciones:
\begin{enumerate}

\item El usuario inicia explícitamente la app.
\item La app actualmente tiene un proceso en primer plano 
(ya sea como actividad o como servicio en primer plano, o en uso por
 parte de otra actividad u otro servicio en primer plano).
\item La app genera una notificación que los usuarios ven en la pantalla 
bloqueada o en la bandeja de notificaciones.
\end{enumerate}

Cuando el usuario enchufa el dispositivo en una fuente de energía,
 el sistema libera las apps del estado de reposo, con lo cual les permite
 acceder libremente a la red y ejecutar cualquier tarea y sincronización
 pendientes. Si el dispositivo queda inactivo durante períodos prolongados, 
las apps inactivas pueden acceder a la red aproximadamente una vez al día.

\paragraph{Firebase Cloud Messaging}

Para poder utilizar nuestra aplicación en ambos modos de ahorro 
de batería utilizaremos la solución de mensajería 
 Firebase Cloud Messaging~\cite{FIREBASECLOUDMESS}, la cual nos
 permitirá enviar mensajes de forma segura al backend aún estando en 
ahorro de energía. Nos resultará sencilla su implementación al ya tener 
Firebase integrado en nuestra aplicación.
