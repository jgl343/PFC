\subsubsection{Firebase Realtime Database}
Firebase Realtime Database es una base de datos alojada en la nube
 que nos permitirá almacenar y sincronizar datos. Los datos se 
sincronizan con todos los clientes en tiempo real y se mantienen
 disponibles cuando la app no tiene conexión.
\\

Los datos se almacenan en formato JSON y se sincronizan en tiempo real
 con cada cliente conectado. Cuando compilas apps multiplataforma con 
el SDK de iOS, Android y JavaScript, todos los clientes comparten una
 instancia de Realtime Database y reciben actualizaciones automáticamente
 con los datos más recientes.
\\

Las funciones clave que nos proporciona son las siguientes:

\begin{enumerate}
\item \textbf{Tiempo real:}
En lugar de solicitudes típicas de HTTP, Firebase Realtime Database 
usa la sincronización de datos (cada vez que cambian los datos,
 los dispositivos conectados reciben esa actualización en milisegundos).
 Proporciona experiencias colaborativas y envolventes sin pensar en el 
código de red.
\item \textbf{Sin conexión:}
Las apps Firebase continúan respondiendo, incluso sin conexión,
 dado que el SDK de Firebase Realtime Database hace que tus datos
 persistan en el disco. Cuando se restablece la conexión, el dispositivo 
cliente recibe los cambios que faltaban y los sincroniza con el estado 
actual del servidor.
\item \textbf{Acceso desde dispositivos cliente:}
Se puede acceder a Firebase Realtime Database directamente desde 
un dispositivo móvil o un navegador web; no se necesita un servidor 
de aplicaciones. La seguridad y la validación de datos están disponibles 
a través de las reglas de seguridad de Firebase Realtime Database: 
reglas basadas en expresiones que se ejecutan cuando se leen o se 
escriben datos.
\item \textbf{Escalamiento en varias bases de datos:}
Con Firebase Realtime Database, puedes satisfacer las necesidades
 de datos de la app a gran escala: podrás dividir la información en 
diversas instancias de bases de datos dentro del mismo proyecto de
 Firebase. Usa Firebase Authentication para optimizar el proceso de 
autenticación en el proyecto. Podrás autenticar a usuarios en varias
 instancias de la base de datos. Controla el acceso a la información de
 cada base de datos. Para ello, usa las reglas personalizadas de
 Firebase Realtime Database en cada una de las instancias de la
 base de datos.

\end{enumerate}

\paragraph{¿Cómo funciona?}

Firebase Realtime Database te permite compilar aplicaciones ricas 
y colaborativas, ya que permite el acceso seguro a la base de datos
 directamente desde el código del cliente. Los datos persisten de 
forma local. Además, incluso cuando no hay conexión, se siguen 
activando los eventos en tiempo real, lo que proporciona una 
experiencia adaptable al usuario final. Cuando el dispositivo
 vuelve a conectarse, Realtime Database sincroniza los cambios de
 los datos locales con las actualizaciones remotas que ocurrieron
 mientras el cliente estuvo sin conexión, lo que combina los conflictos
 de forma automática.

Realtime Database proporciona un lenguaje flexible de reglas
 basadas en expresiones, llamado reglas de seguridad de Firebase
 Realtime Database, para definir cómo se deberían estructurar los
 datos y en qué momento se pueden leer o escribir. Integrar 
Firebase Authentication permite que los programadores definan 
quién tiene acceso a qué datos y cómo acceden a ellos.

Realtime Database es una base de datos NoSQL y, como tal,
 tiene diferentes optimizaciones y funcionalidades en comparación
 con una base de datos relacional. La API de Realtime Database 
está diseñada para permitir solo operaciones que se puedan
 ejecutar rápidamente. Eso permite crear una excelente 
experiencia de tiempo real que puede servir a millones de usuarios
 sin afectar la capacidad de respuesta. Es importante pensar 
cómo deben acceder a los datos los usuarios y estructurarlos
 según corresponda.