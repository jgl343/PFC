\subsubsection{Bases de datos no relacionales NoSQL}

Empezamos definiendo el concepto de base de datos no relacionales
 NoSQL.

Pese a la no existencia de una definición formal, cuando hablamos 
de base datos NoSQL nos referimos a una amplia clase de sistemas
 de gestión de datos (mecanismos para el almacenamiento y
 recuperación de datos) que difieren, en aspectos importantes, 
del modelo clásico de relaciones entre entidades (o tablas) existente
 en los sistemas de gestión bases de datos relacionales, siendo el más 
destacado el que no usan SQL como lenguaje principal de consulta.
Aunque son conocidas desde la década de los 60 del pasado siglo,
 su auge actual viene determinado por el uso que, de estos sistemas
 han hecho las principales compañías de internet como Amazon,
 Google, Twitter y Facebook. Estas compañías tenían que enfrentarse
 a nuevos desafíos en el tratamiento de los datos motivados por
 el enorme crecimiento de la Web donde se requería dar respuesta
 a la necesidad de proporcionar información procesada a partir de
 grandes volúmenes de datos con unas estructuras horizontales,
 más o menos, similares y con aplicaciones web que debían dar 
respuesta a las peticiones de un número elevado e indeterminado
 de usuarios en el menor tiempo posible. Estas compañías se dieron 
cuenta de que el rendimiento y sus necesidades de tiempo real eran
 más importantes que la consistencia de los datos, aspecto este último
 al que las bases de datos relacionales tradicionales dedicaban 
una gran cantidad de tiempo de proceso.

Las características comunes entre todas las implementaciones de
 bases de datos NoSQL suelen ser las siguientes:
\begin{enumerate}
\item \textbf{Consistencia Eventual:} A diferencia de las bases 
de datos relacionales tradicionales, en la mayoría de sistemas NoSQL,
 no se implementan mecanismos rígidos de consistencia que garanticen
 que cualquier cambio llevado a cabo en el sistema distribuido sea
 visto, al mismo tiempo, por todos los nodos y asegurando, también, la
 no violación de posibles restricciones de integridad de los datos u
 otras reglas definidas. En su lugar y para obtener un mayor
 rendimiento, se ofrece el concepto de ``consistencia eventual'', en
 el que los cambios realizados ``con el tiempo'' serán propagados a
 todos los nodos por lo que, una consulta podría no devolver los
 últimos datos disponibles o proporcionar datos inexactos, problema
 conocido como lecturas sucias u obsoletas.  Asimismo, en algunos
 sistemas NoSQL\footnote{Fuente: Oracle~\cite{DEFINICIONNOSQL} y NoSQL
 for mere mortals~\cite{NOSQL}} se pueden presentar perdidas de datos
 en escritura. Esto se conoce también como BASE (Basically Available
 Soft-state Eventual Consistency), en contraposición a ACID
 (Atomicity, Consistency, Isolation, Durability), su analogía en las
 bases de datos relacionales.
\item \textbf{Flexibilidad en el esquema:} En la mayoría de
 base de datos NoSQL, los esquemas de datos son dinámicos; 
es decir, a diferencia de las bases de datos relacionales en las que,
 la escritura de los datos debe adaptarse a unas estructuras(o tablas, 
compuestas a su vez por filas y columnas) y tipos de datos pre-definidos,
 en los sistemas NoSQL, cada registro (o documento, como se les 
suele llamar en estos casos) puede contener una información con 
diferente forma cada vez, pudiendo así almacenar sólo los atributos 
que interesen en cada uno de ellos, facilitando el polimorfismo de 
datos bajo una misma colección de información. También se pueden 
almacenar estructuras complejas de datos en un sólo documento, 
como por ejemplo almacenar la información sobre una publicación 
de un blog (título, cuerpo de texto, autor, etc) junto a los comentarios
 y etiquetas vertidos sobre el mismo, todo en un único registro.
\item \textbf{Escalabilidad horizontal:} Por escalabilidad horizontal
 se entiende la posibilidad de incrementar el rendimiento del sistema
 añadiendo, simplemente, más nodos (servidores) e indicando al sistema
 cuáles son los nodos disponibles.
\item \textbf{Estructura distribuida:} Generalmente los datos se 
distribuyen, entre los diferentes nodos que componen el sistema. 
Hay dos estilos de distribución de datos:
\begin{enumerate}
\item \textbf{Particionado (Sharding):} El particionado distribuye 
los datos entre múltiples servidores de forma que, cada servidor,
 actúe como única fuente de un subconjunto de datos. Normalmente, 
a la hora de realizar esta distribución, se utilizan mecanismos de tablas
 de hash distribuidas (DHT). 
\item \textbf{Réplica:} La réplica copia los datos entre múltiples 
servidores, de forma que cada bit de datos pueda ser encontrado 
en múltiples lugares. Esta réplica puede realizarse de dos maneras:
Réplica maestro-esclavo en la que un servidor gestiona la escritura
 de la copia autorizada mientras que los esclavos se sincronizan con
 este servidor maestro y sólo gestionan las lecturas.
Réplica peer-to-peer en la que se permiten escrituras a cualquier nodo 
y ellos se coordinan entre sí para sincronizar sus copias de los datos.
\end{enumerate}
\item \textbf{Tolerancia a fallos y Redundancia: } Pese a lo que 
cualquiera pueda pensar cuando se habla de NoSQL, no todas las 
tecnologías existentes bajo este paraguas usan el mismo modelo de
 datos ya que, al ser sistemas altamente especializados, la idoneidad 
particular de una base de datos NoSQL dependerá del problema
 a resolver. Así a todo, podemos agrupar los diferentes modelos de
 datos usados en sistemas NoSQL en cuatro grandes categorías:
\begin{enumerate}
\item \textbf{Base de datos de Documentos: } Este tipo de base 
de datos almacena la información como un documento, usando para
 habitualmente para ello una estructura simple como JSON, BSON o XML
 y donde se utiliza una clave única para cada registro. Este tipo de
 implementación permite, además de realizar búsquedas por clave-valor,
 realizar consultas más avanzadas sobre el contenido del documento.
 Son las bases de datos NoSQL más versátiles.
\item \textbf{Almacenamiento Clave-Valor:} Son el modelo de base de
 datos NoSQL más popular, además de ser la más sencilla en cuanto a 
funcionalidad. En este tipo de sistema, cada elemento está identificado 
por una clave única, lo que permite la recuperación de la información de
 forma muy rápida, información que suele almacenarse como un objeto
 binario. Se caracterizan por ser muy eficientes tanto para las lecturas
 como para las escrituras. 
\item \textbf{Bases de datos de grafos:} Usadas para aquellos datos 
cuyas relaciones se pueden representar adecuadamente mediante 
un grafo. Los datos se almacenan en estructuras grafo con nodos 
(entidades), propiedades (información entre entidades) y líneas (conexiones
 entre las entidades). 
\item \textbf{Base de datos Columnar (o Columna ancha):} En vez de
 tablas, en las bases de datos de columna tenemos familias de columnas
 que, son los contenedores de las filas. A diferencia de los RDBMS, 
no necesita conocer de antemano todas las columnas, cada fila no tiene 
por qué tener el mismo número de columnas. Este tipo de bases de datos
 se adecuan mejor a operaciones analíticas sobre grandes conjuntos de datos.
\end{enumerate}

\end{enumerate}
Pese a todas las opciones proporcionadas por el auge de las bases
 de datos NoSQL, esto no significa la desaparición de las bases de 
datos de RDBMS ya que son tecnologías complementarias. Estamos
 entrando en una era de persistencia políglota, una técnica que utiliza 
diferentes tecnologías de almacenamiento de datos para manejar las 
diversas necesidades de almacenamiento de datos. Hemos elegido la 
implementación de una base de datos NoSQL para nuestra aplicación
 ya que para nosotros el rendimiento es más importante que preservar 
el modelo relacional. Además de la posibilidad de escalar de forma sencilla, 
hemos tenido en cuenta otros criterios: 
\begin{enumerate}
\item El volumen de lecturas y escrituras.
\item Tolerancia por datos inconsistentes en réplicas.
\item La naturaleza de las relaciones entre entidades y cómo eso afecta
 los patrones de consulta.
\item Disponibilidad y requisitos de recuperación de desastres.
\item La necesidad de flexibilidad en los modelos de datos.
\item Requisitos de latencia.
\end{enumerate}
\subsubsection{Bases de datos no relacionales NoSQL}
Nos vamos a decantar por un modelo Almacenamiento Clave-Valor
 y vamos a almacenar una lista de estructuras JSON.
\\
Este tipo de bases de datos Clave-Valor se utilizan en una amplia gama
 de aplicaciones, como las siguientes:
\begin{enumerate}
\item Almacenamiento en caché de datos de bases de datos 
relacionales para mejorar el rendimiento.
\item Seguimiento de atributos transitorios en una aplicación web, 
como un carrito de compras.
\item Almacenamiento de configuración e información de datos de usuario
 para aplicaciones móviles.
\item Almacenar objetos grandes, como imágenes y archivos de audio.
\end{enumerate}

