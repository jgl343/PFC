\subsection{Firebase}
Firebase es una plataforma para el desarrollo de aplicaciones web y
 aplicaciones móviles. Firebase se integra actualmente con otros
 servicios de Google para poder ofrecer productos a mayor escala
 para los desarrolladores.
Los servicios ofrecidos por Firebase son los siguientes: 
\begin{enumerate}
\item \textbf{Firebase Analytics: } es una aplicación
 gratuita que proporciona una visión profunda sobre el
 uso de la aplicación por parte de los usuarios.
\item \textbf{Firebase Cloud Messaging: }
Antiguamente conocido como Google Cloud Messaging (GCM),
 Firebase Cloud Messaging (FCM) es una plataforma para mensajes 
y notificaciones para Android, iOS, y aplicaciones web que
 actualmente puede ser usada de forma gratuita.
\item \textbf{Firebase Auth:} es un servicio que puede autenticar
 los usuarios utilizando únicamente código del lado del cliente.
 Incluye la autenticación mediante Facebook, GitHub, Twitter y 
Google. Además, incluye un sistema de administración del usuario
 por el cual los desarrolladores pueden habilitar la autenticación de
 usuarios con email y contraseña que se almacenarán en Firebase.
\item \textbf{Realtime Database:}
Firebase proporciona una base de datos en tiempo real y back-end.
 El servicio proporciona a los desarrolladores de aplicaciones una API
 que permite que la información de las aplicaciones sea sincronizada
 y almacenada en la nube de Firebase. La compañía habilita integración
 con aplicaciones Android, iOS, JavaScript, Java, Objective-C, Swift
 y Node.js. La base de datos es también accesible a través de una 
REST API e integración para varios sistemas de Javascript como
 AngularJS, React, Ember.js y Backbone.js.La REST API utiliza el 
protocolo SSE (del inglés Server-Sent Events), el cual es una API
 para crear conexiones de HTTP para recibir notificaciones push 
de un servidor.
\item \textbf{Firebase Storage:} proporciona cargas y descargas
 seguras de archivos para aplicaciones Firebase, sin importar la 
calidad de la red. El desarrollador lo puede utilizar para almacenar
 imágenes, audio, vídeo, o cualquier otro contenido generado por
 el usuario. Firebase Storage se basa en el almacenamiento de
 Google Cloud Storage.
\item \textbf{Firebase Firestore:} es un servicio derivado de 
Google Cloud Platform, adaptado a la plataforma de Firebase.
 Al igual que Realtime Database, es una base de datos NoSQL,
 aunque presenta diversas diferencias. Se organiza en forma 
de documentos agrupados en colecciones, y en ellos se pueden
 incluir tanto campos de diversos tipos (cadenas de texto, números,
 puntos geográficos, referencias a la propia base de datos, arrays,
 booleanos, marcas de tiempo, e incluso objetos propios) como 
otras subcolecciones.
\end{enumerate}

Utilizaremos Realtime Database ya que nos ofrece la estabilidad 
de un producto muy probado y usado y muy baja latencia, por lo
 que es una excelente opción para la sincronización frecuente de 
estados.