\subsection{JSON}
\label{json}
JSON (JavaScript Object Notation - Notación de Objetos de JavaScript)~\cite{JSON} 
es un formato ligero de intercambio de datos. Leerlo y escribirlo es simple
 para humanos, mientras que para las máquinas es simple interpretarlo y 
generarlo. Está basado en un subconjunto del Lenguaje de Programación
 JavaScript. JSON es un formato de texto que es completamente independiente
 del lenguaje pero utiliza convenciones que son ampliamente conocidos por los
 programadores de la familia de lenguajes C, incluyendo C, C++, Java, 
JavaScript, Perl, Python, y muchos otros. Estas propiedades hacen que 
JSON sea un lenguaje ideal para el intercambio de datos.

JSON está constituído por dos estructuras:
\begin{enumerate}

\item Una colección de pares de nombre/valor. En varios lenguajes 
esto es conocido como un objeto, registro, estructura, diccionario, 
tabla hash, lista de claves o un arreglo asociativo.
\item Una lista ordenada de valores. En la mayoría de los lenguajes, 
esto se implementa como arreglos, vectores, listas o sequencias.
\end{enumerate}
Estas son estructuras universales; virtualmente todos los lenguajes de
programación las soportan de una forma u otra. Es razonable que un
formato de intercambio de datos que es independiente del lenguaje de
programación se base en estas estructuras.
