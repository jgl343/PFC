\subsection{Java}
Java es un lenguaje de programación orientado a objetos de propósito
 general, concurrente,  orientado a objetos que fue diseñado
 específicamente para tener tan pocas dependencias de implementación
 como fuera posible. Su intención es permitir que los desarrolladores
 de aplicaciones escriban el programa una vez y lo ejecuten en 
cualquier dispositivo (conocido en inglés como WORA, o 
``write once, run anywhere''), lo que quiere decir que el código
 que es ejecutado en una plataforma no tiene que ser compilado de
 nuevo para correr en otra. Java es, a partir de 2012, uno de los
 lenguajes de programación más populares en uso, particularmente
 para aplicaciones de cliente-servidor de web, con unos 10 millones
 de usuarios reportados~\cite{JAVA}.
\subsubsection{Mecanismos básicos}
\begin{enumerate}
\item \textbf{Objetos:}
 Es una abstracción encapsulada genérica de datos y los procedimientos
 para manipularlos.
\item \textbf{Mensajes:}
Los objetos se comunican a través de señales o mensajes, siendo estos
 mensajes los que hacen que los objetos respondan de diferentes maneras.
\item \textbf{Métodos:}
Es una acción que determina como debe de actuar un objeto cuando 
recibe un mensaje.
\item \textbf{Clases:}
Es la generalización de un tipo específico de objetos, es decir, es el 
conjunto de características y comportamientos de todos los objetos 
que componen la clase.
\end{enumerate}
\subsubsection{Principios fundamentales}
\begin{enumerate}
\item \textbf{Abstracción:}
Es el proceso de representar entidades reales como elementos
 internos a un programa, la abstracción de los datos es tanto en los atributos, 
como en los métodos de los objetos.
\item \textbf{Encapsulamiento:}
Cada objeto está aislado del exterior, esta característica permite verlo 
como una caja negra, que contiene toda la información relacionada con ese
 objeto. Este aislamiento protege a los datos asociados a un objeto para
 que no se puedan modificar por quien no tenga derecho a acceder a ellos.
Permite manejar a los objetos como unidades básicas, dejando oculta su
 estructura interna.
\item \textbf{Herencia:}
Es el medio para compartir en forma automática los métodos y los datos 
entre las clases y subclases de los objetos. Los objetos heredan las
 propiedades y el comportamiento de todas las clases a las que pertenece.
\item \textbf{Polimorfismo:}
Esta característica facilita la implementación de varias formas de un mismo método,
 con lo cual se puede acceder a varios métodos distintos, que tienen el mismo
 nombre~\cite{JAVA}.
\end{enumerate}
\subsubsection{Entornos de funcionamiento}
	
El diseño de Java, su robustez de la industria y su fácil portabilidad han 
hecho de Java uno de los lenguajes con un mayor crecimiento y amplitud de
 uso en dispositivos ámbitos de la industria de la informática. Los distintos
 entornos donde podemos ejecutar una aplicación Java son:
\begin{enumerate}
\item	En dispositivos móviles y sistemas embebidos.
\item	En el navegador web.
\item	En sistemas de servidor.
\item	En aplicaciones de escritorio.
\end{enumerate}
