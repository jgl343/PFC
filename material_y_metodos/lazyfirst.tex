
Hacer una aplicación Lazy First significa encontrar formas de reducir y optimizar operaciones que supongan una carga significativa para la carga de la batería. Las conceptos que deben aplicarse son los siguientes:
\begin{enumerate}
\item \textbf{Reducir:} Reducir el número de operaciones que se realizan, por ejemplo, utilizar una cache de datos descargados, en lugar de volver a descargarlos cada vez que los necesitemos.
\item \textbf{Aplazar:} Aplazar operaciones hasta que no supongan un impacto en la duración de la batería, por ejemplo, esperar a tener el cargador conectado para subir información al servidor.
\item \textbf{Agrupar:} Agrupar varias operaciones que puedan para que al realizarse de forma simultanea nos ahorre, por ejemplo, sacar el dispositivo de reposo varias veces.
\end{enumerate}

Deberemos utilizar estos principios cada vez que queramos utilizar la CPU, la antena, la pantalla o el GPS del dispositivo.