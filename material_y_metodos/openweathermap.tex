\subsection{OpenWeather Weather API}

OpenWeather nos ofrece una potente API con la que obtendremos el pronóstico meteorológico diario. Con una simple llamada como la siguiente:
\begin{itemize}


\item  \url{http://api.openweathermap.org/data/2.5/weather?lat=36.7311957&lon=-2.854627&appid=d488064b89a68d822fd0f952df6b7c83}
\end{itemize}
Podremos obtener la información necesaria acerca de las condiciones meteorológicas. Los parámetros que deberemos rellenar son los siguientes:

\begin{enumerate}
\item \textbf{lat:} La latitud será la coordenada geográfica proporcionada por el sistema GPS de nuestro dispositivo móvil, nos servirá para identificar la posición. En el ejemplo es \emph{36.7311957}.
\item \textbf{lon:} La longitud será la Coordenada geográfica proporcionada por el sistema GPS de nuestro dispositivo móvil, nos servirá para identificar la posición. En el ejemplo es \emph{-2.854627}.
\item \textbf{appid:} Nuestra API Key, necesaria para autentificar la llamada, nos la proporcionan al crearnos una cuenta en OpenWeatherMap \cite{OPENWEATHER}. En el ejemplo es \emph{d488064b89a68d822fd0f952df6b7c83}.
\end{enumerate}

La respuesta que recibiremos al hacer la llamada del ejemplo, será un mensaje JSON, con la siguiente estructura:
\\
\lstset{
    string=[s]{"}{"},
    stringstyle=\color{blue},
    comment=[l]{:},
    commentstyle=\color{black},
}
\begin{lstlisting}
{
"coord":
	{
		"lon":-2.85,
		"lat":36.73
	},
"weather":[
	{
		"id":800,
		"main":"Clear",
		"description":"clear sky",
		"icon":"01n"
	}
		],
"base":"stations",
"main":
	{
		"temp":285.514,
		"pressure":1013.75,
		"humidity":100,
		"temp_min":285.514,
		"temp_max":285.514
	},
"wind":
	{
		"speed":5.52,
		"deg":311
	},
"clouds":
	{
		"all":0
	},
"dt":1533231000,
"sys":
	{
		"message":0.0025,
		"country":"ES",
		"sunrise":1533187107,
		"sunset":1533237375
	},
"id":2521382,
"name":"Balerma",
"cod":200
	}
}
\end{lstlisting}

A continuación vamos a describir cada uno de los atributos de la respuesta:

\begin{enumerate}
\item \textbf{coord:}
\begin{enumerate}
\item lon: Longitud.
\item lat: Latitud.
\end{enumerate}
\item \textbf{weather:} Códigos de la información meteorológica.
\begin{enumerate}
\item id: Identificador del clima.
\item main: Grupo de condición meteorológica (Lluvia, Nieve, Despejado, etc.)
\item description: Condición meteorológica dentro de su grupo
\item  icon. Icono del clima.
\end{enumerate}
\item  \textbf{base:} Párametro interno.
\item  \textbf{main:}
\begin{enumerate}
\item  temp: Temperatura. Unidad por defecto: Kelvin.
\item  pressure: Presión atmosférica.
\item  humidity: Porcentaje de humedad.
\item  temp\_min: Temperatura mínima en el momento. Unidad por defecto: Kelvin.
\item  temp\_max: Temperatura máxima en el momento. Unidad por defecto: Kelvin.
\end{enumerate}
\item  \textbf{wind:}
\begin{enumerate}
\item speed: Velocidad del viento. Unidad por defecto: m/s.
\item deg: Dirección del viento, en grados.
\end{enumerate}
\item  \textbf{clouds:}
\begin{enumerate}
\item all: Porcentaje de nubosidad.
\end{enumerate}
\item  \textbf{dt:}Tiempo empleado en calcular los datos, unix, UTC.
\item  \textbf{sys:}
\begin{enumerate}
\item type: Parámetro interno.
\item id: Parámetro interno.
\item message: Parámetro interno.
\item country: Código del país (GB, JP etc.)
\item sunrise: Hora del amanecer, unix, UTC.
\item sunset: Hora del anochecer, unix, UTC.
\end{enumerate}
\item  \textbf{id:} Identificador de la población.
\item  \textbf{name:} Nombre de la población.
\item  \textbf{cod:} Parámetro interno.
\end{enumerate}

Tendremos especialmente en cuenta los parámetros de la velocidad y la dirección del viento, ya que serán los que utilicemos para comprobar si el usuario de la aplicación está a la deriva y realizar la alerta de rescate.

\footnote{Fuente: Open Weather \cite{OPENWEATHER}}
