\chapter{Revisión bibliográfica}
\label{revision}

Los nuevos tiempos cambian y con ella los instrumentos usados en la
navegación sobre todo desde que la electrónica irrumpió de forma
intensiva.

Nos tenemos que remontar hasta la aparición de la normativa 
NMEA(National Marine Electronics Asociation),la cual permitía
concentrar información dispersa en una sola pantalla procedente de
cada uno de los componentes de la electrónica de a bordo (corredera,
equipo de viento, sonda, radar, posición), sin importar su fabricante
o el modelo.

Posteriormente la aparición de otros elementos como el AIS (Automatic
Identification System), GPS y plotters de alta definición nos han
llevado a recurrir a nuevas herramientas y dispositivos para presentar
toda la información de ayuda a la navegación a través de un PC con
programas aplicaciones de cartografía escaneada o vectorizada (como
OziExplorer, Navionics o Tsunamis), que nos mostraban gracias al
estándar NMEA, toda la información que teníamos distribuida en los
diversos dispositivos electrónicos instalados a bordo.

Actualmente la mayoría de los navegantes han pasado del papel a las
cartas leídas en el ordenador (y ,del compás al ratón, y de la
corredera al GPS), pero aun toca asimilar el siguiente paso :integrar
todas las herramientas de ayuda a la navegación en nuestros
dispositivos móviles .

Hoy en día los sistemas de geolocalización ayudados de una conexión a
Internet han revolucionado el transporte, de modo que con un
smartphone o tableta podemos tener a nuestro alcance un
multi-dispositivos a muy bajo coste con toda la información que hace
unos años sólo podíamos obtener usando electrónica bastante compleja y
costosa. De este modo, con la evolución y desarrollo de las tabletas
Android, las necesidades de electrónica y su utilidad en las
embarcaciones ha supuesto un cambio en cuanto a los hábitos de muchos
navegantes, ganando cada vez más importancia la tableta como el
elemento principal para contener la información que se necesita para
la navegación.

Entre las aplicaciones náuticas del mercado que vamos a ver algunas
requieren una conexión a Internet, pero muchas otras ofrecen la opción
de trabajar sin cobertura, usando señales de posicionamiento GPS o de
los sensores internos (acelerómetro o brújula). Veamos una muestra de
algunas de las aplicaciones mas usadas
  
\section{Seguridad}
La app gratuita de Salvamento Marítimo también es indispensable entre
las aplicaciones náuticas.
\begin{enumerate}
\item \textbf{Salvamento Marítimo:} Con esta aplicación móvil
  obtenemos información y orientación sobre cuestiones relacionadas
  con la seguridad náutica. Por ejemplo, nos ofrece la posibilidad de
  realizar una serie de controles para ver si estamos cumpliendo con
  los puntos de seguridad que necesitamos cumplir antes de salir a
  navegar. Desde consejos y todo tipo de información para navegantes,
  hasta listados para controlar los puntos esenciales de seguridad
  antes de la navegación.

  Esta aplicación permite a los navegantes de recreo que Salvamento
  Marítimo monitorice su travesía y que de la alarma en caso de no
  cumplir con el plan de navegación establecido.

  El 50\% de las incidencias que atiende Salvamento Marítimo, son de
  embarcaciones de recreo.

\item \textbf{SafeTrx:} SafeTrx es una aplicación para Smartphone,
  tanto para dispositivos Android como Apple iOS (iPhone, iPad), que
  monitoriza los viajes de su embarcación, avisa a los contactos
  designados por el usuario cuando hay retrasos en el viaje
  planificado y ofrece una página web para que Salvamento Marítimo
  pueda consultar con rapidez la derrota realizada por la embarcación
  y tomar las acciones oportunas.

  Esta aplicación complementa al Sistema Mundial de Socorro y
  Seguridad Marítima.
\end{enumerate}

\section{Aplicaciones de AIS}
En los últimos años hemos visto cómo han aparecido algunas
aplicaciones que nos ayudan a identificar las embarcaciones. Se trata
de aplicaciones que usan el AIS (Automatic Identification
System). Gracias a este sistema podremos conocer la identidad, el
rumbo, la posición y más datos sobre aquellas embarcaciones que estén
situadas en la zona próxima a la que nos encontramos nosotros.

Habitualmente esto se había realizado a través de sistemas de radio
VHF, pero actualmente gracias a una conexión a internet se puede
reemplazar su cometido con aplicaciones como MarineTrafic,

\begin{enumerate}
\item \textbf{mAIS marine Traffic Ship Position Reporting:} Esta App
  basada en Google Maps, nos permiten enviar y visualizar información
  a tiempo real de la señal AIS. Para ello tan solo necesitamos tener
  acceso a una conexión de internet y gracias al soporte de Google
  Maps, pueden visualizarse los barcos en cualquier lugar del
  mundo. Para emitir la señal es preciso tener una conexión a
  Internet, esta emisión de señal hay que pararla al llegar a tierra.
\item \textbf{Ship Finder – Live vessel tracking:} Con esta App
  también puede obtenerse la información sobre todas las embarcaciones
  situadas en la zona seleccionada. Hay una versión gratuita y otra
  más completa de pago.
\end{enumerate}

\section{Aplicaciones náuticas meteorológicas}

Todos los navegantes están pendientes de la previsión
meteorológica. Hoy en día tenemos acceso a datos a tiempo real sobre
la evolución de los indicadores básicos en meteorología, así como a
los pronósticos. En el universo de las aplicaciones encontramos una
gran cantidad de herramientas que nos ayudarán a conocer las
condiciones de viento y previsión de lluvia con gran lujo de detalles.

\begin{enumerate}
\item \textbf{WeatherPRO HD:} es una aplicación realizada por
  Meteogrup que ofrece un pronóstico a una semana vista, en intervalos
  de tres horas, en más de dos millones de lugares, con radar en
  tiempo real e imágenes por satélite.
\item \textbf{WinFinderPRO:} para los aficionados a los deportes de
  vela es una aplicación fundamental. Viento, olas y predicción
  meteorológica son los principales reclamos de esta app. Su uso es
  bastante parecido a la WeatherPRO, con la posibilidad de tener
  nuestra lista de favoritos.
\item \textbf{PocketGrib:} Esta app hace posible visualizar todos los
  datos meteorológicos en una previsión de hasta ocho días, con un
  elevado nivel de acierto en las previsiones de viento. Basada en su
  hermano mayor diseñado para PC, uGRib (www.grib.us) permite ver
  todos los datos meteorológicos con una previsión de hasta ocho
  días. Seleccionamos directamente en la pantalla la zona sobre la que
  queremos obtener la previsión, y nos muestra de forma gráfica las
  previsiones de viento con una alta precisión y un alto grado de
  acierto.
\item \textbf{WindGuru:} Se trata de una aplicación muy conocida para
  los navegantes entre las aplicaciones náuticas, que nos muestra de
  forma gráfica y muy rápida las previsiones de viento y temperatura
  en la zona que seleccionemos, con un alto grado de fiabilidad.
\item \textbf{Eltiempo.es:} La conocida página de internet de Jose
  Antonio Maldonado tiene su APP, con una visualización panorámica,
  hora a hora, de vientos, oleajes y temperaturas en la zona de
  navegación que se seleccione, con unos datos gráficos de gran
  exactitud.
\item \textbf{Rain Alarm:} Mediante esta APP, pueden recibirse los
  avisos correspondientes mediante alertas, de aquellas lluvias que se
  acercan, con imágenes animadas sobre la meteorología a nivel
  mundial.
\end{enumerate}

\section{Aplicaciones náuticas de geolocalización}
\begin{enumerate}
\item \textbf{Boating, aplicación de Navionics:} es la app náutica de
  cartografía más vendida en el mundo, y ofrece todas las
  posibilidades de planificar rutas con facilidad y comodidad, y en
  compatibilidad con Google Earth y otras aplicaciones de
  meteorología. Existe una versión HD con más funciones. Se puede usar
  sin conexión a internet, si previamente a descargado las cartas.

\item \textbf{Navionics HD:} Navionics es el software de cartografía
  más demandado y vendido en todo el mundo. Ha tenido una gran
  evolución desde su origen, permitiendo utilizar la tableta como un
  auténtico ploter, con las ventajas de poder moverlo por todo el
  barco e incluso planificar rutas o ver la información desde
  cualquier sitio. La versión HD tiene una precisión impresionante y
  se integra con Google Earth y con ficheros Grib de meteorología,
  además de información en tiempo real de servicios cercanos a nuestra
  ubicación (marinas, mecánicos, restaurantes…). Su uso es muy
  sencillo y no requiere nada más que instalar la App y comenzar a
  navegar. Puede trabajar sin conexión a internet, cargando
  previamente las cartas.

\item \textbf{Google Earth:} La aplicación del gigante de las
  búsquedas en Internet no puede faltar. Con toda precisión se puede
  visualizar la ruta a seguir, así como fotos precisas del
  lugar. Algunas de las demás App se integran con Google earth. No
  puede faltar.
\item \textbf{Gabenative:} Es otra aplicación náutica que muchos
  aficionados a la navegación suelen tener instalada, y es realmente
  útil para ayudarnos a situar en el medio del mar.
\end{enumerate}

\section{Instrumentación}

\begin{enumerate}
\item \textbf{Polaris Navegación GPS:} Convierte un teléfono en un
  potente sistema de navegación GPS de uso general con Polaris
  Navigation GPS. Utilizado principalmente como una aplicación de GPS
  de náutica y rastro, Polaris es un excelente respaldo o reemplazo
  para su Garmin u otra unidad GPS portátil de navegación por
  satélite.

\item \textbf{Antigarreo:} Una herramienta para ayudarnos en el
  fondeo,una de las variables que más influye en la serenidad del
  barco es el garreo. Con esta aplicación, el navegante recibirá las
  alertas necesarias para que la embarcación no salga de su espacio de
  borneo.

\item \textbf{Bearing Pilot:} Se trata de una aplicación para evitar
  las demoras en la navegación, prever rumbos y eliminar el riesgo de
  colisiones, cálculos de rumbos de viento, mantenimiento de la
  dirección, etc. Una nueva herramienta dirigida tanto a navegantes
  principiantes como a los más profesionales, su extremada sencillez
  nos permitirá tomar marcaciones de forma rápida, mostrándonos en su
  interfaz gráfica la dirección de la demora que hemos tomado, el
  ángulo respecto a nuestro rumbo y también nos indicará hacia que
  banda debemos caer para poner proa a dicha demora.

\item \textbf{Vaavud:} Este app usa un dispositivo de plástico
  resistente al agua, polvo y a la arena. Se conecta al jack de sonido
  para medir la velocidad del viento de una forma precisa en todas las
  direcciones, promedios, velocidades actuales, etc. Vaavud se compra
  por internet directamente en la página del fabricante (vaavud.com)
  por unos 40 Euros. Permite medir de forma muy precisa y fiable la
  velocidad del viento en cualquier dirección, convirtiendo nuestro
  dispositivo iPhone o Android en una pequeña estación
  meteorológica.Además, utilizando la App gratuita de Vaavud, podemos
  ver las mediciones de la velocidad del viento de otros usuarios
  registrados en otras ubicaciones.El dispositivo puede medir con
  precisión la velocidad del viento en un rango de 1 m/s hasta 25 m/s,
  y ha sido probado y calibrado profesionalmente en un túnel de viento
  en la Universidad Técnica de Dinamarca.

\item \textbf{NMEA Remote:} Esta App funciona a través de un módem
  wifi puerto serie que hay que instalar previamente, y actúa como un
  repetidor de datos de la embarcación.
\end{enumerate}
 
\section{Otras aplicaciones}
\begin{enumerate}

\item \textbf{Marinus Ripa:} Marinus Ripa aporta el Reglamento
  Internacional de Prevención de Abordajes, una actualización
  permanente, y toda la información básica de seguridad que un buen
  navegante debe conocer. Además, se puede obtener también una versión
  en iBook para iPad en formato libro de alta definición.

\item \textbf{MoonPreview:} Con esta app es posible conocer el ciclo
  lunar y el estado de la luna la noche que está planificado el fondeo
  de la embarcación. Ideal para viajes de recreo y contemplación de la
  cúpula celeste.

\item \textbf{Useful knots:} Para cada necesidad hay una variedad de
  nudo distinta. Esta App nos enseña y recuerda, paso a paso, la
  correcta forma de realizar hasta cien nudos diferentes, según sus
  tipos y utilidades.

\item \textbf{My112:} Esta aplicación permite comunicarte con el
  Centro 112 de Emergencias, enviando tu posición actual al operador
  que te está atendiendo, ayudando en tu localización. Además, My112
  recibe en tiempo real avisos de emergencias cercanas a tu posición e
  información actualizada de las mismas en el momento de producirse.
\end{enumerate}

\section{Conclusión}

Como podemos ver en el análisis previo de distintas aplicaciones, hay
muchas cuya función es el posicionamiento marítimo, la obtención de
información meteorológica y que ofrecen seguridad en forma de avisos o
llamadas de emergencia. Hemos realizado una búsqueda pero no hemos
logrado encontrar ninguna que unifique los criterios de malas
condiciones meteorológicas y alerta temprana de situaciones de riesgo,
que es el objetivo que pretendemos conseguir, lo que creemos que
podría hacerse un hueco en el saturado mercado de las aplicaciones
móviles.
